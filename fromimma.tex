QF-PCR is based on the assumption that within the exponential phase of PCR amplification, the
amount of specific DNA produced is proportional to the quantity of the initial target. In order to achieve this the optimal number of PCR cycles must be limited to avoid reaching an amplification plateau. By amplifying highly polymorphic regions specific for a chromosome, such as short tandem repeats (STR), we should find a high rate of heterozygosity among different allelic forms and individuals. Tetra- or pentanucleotide repeats have been preferentially employed in view of their stability and suitability for amplification and analysis. By labelling primers with a fluorescent dye, we are able to detect dosage ratios of the PCR products from the analysis of the fluorescent peak areas shown by a Genetic Analyzer.


Thus, in normal heterozygotes the ratio of fluorescent activity for the two peaks  corresponding to the PCR products should be within the range 0.8–1.4 (disomic diallelic). Few normal subjects should be homozygotes showing one peak of activity (disomic monoallelic). Besides, in a trisomic patient the three doses of an STR marker can be detected either as three peaks of fluorescent activities with a 1:1:1 ratio (trisomic triallelic) or as a
pattern of two peaks with a ratio or dosage &lt;0.65 or &gt;1.8 (trisomic diallelic) (Hulten et al., 2003)

(Figure 1). Triploidy of specimens is assumed when all of the markers studied, which map to
different chromosomes, show a trisomic pattern of amplification. Uniparental disomy (UPD) for a targeted region or chromosome is assumed when the pattern of amplification of various STR
markers corresponds to the inherited alleles from one progenitor with the absence of the other progenitor alleles.
A rapid screening of sex and numerical anomalies for chromosomes 13, 15, 16, 18, 21, 22 and X
was carried out with the miscarriage DNA samples performing five multiplex PCR assays PCR
assays A, B 1,2 and 3 were performed in a total volume of 25 μl containing 40–100 ng of genomic DNA, 10mM dNTP (Roche), 6-30 pmol (final concentration) of each primer, 1 × Fast taq
polymerase buffer (15 mmol/l MgCl 2 ) (Roche), and 2.5 U of Fasta taq polymerase (Roche). PCR
conditions were as follows: denaturation at 95 °C for 10 min followed by 10 cycles consisting of melting at 95°C for 1 min, annealing at 65°C (-1°C / cycle) for 1 minutes, and then extension at 72°C for 40 seconds, then for 23 cycles at 95 °C for 1 min, 55 °C for 1 min, and 72 °C for 1 min.
Final extension was for 10 min at 72 °C and at 60 °C for 60 min. Fluorescence-labelled PCR
products were electrophoresed in an CEQ 8000 Backman. by combining 40 μl of Hi-Di Formamide
and 0.5 μl of DNA size standard 400 (Backman); The resulting PCR products can be visualized and quantified as peak areas of the respective repeat lengths In normal heterozygous subjects, the QF- PCR product of each STR should show two peaks with similar fluorescent activities and thus a ratio of peak areas close to 1:1 (ranging from 0.8 to 1.4:1). A trisomy is suspected when the ratio is  above or below this range (peak area ratios ≤ 0.6 and ≥ 1.8) (trisomic diallelic);otherwise there are three alleles of equal peak area with a ratio of 1:1:1 (trisomictriallelic). The presence of trisomic triallelic or diallelic patterns for at least two different STRs on the same chromosome is considered as evidence of trisomy.

Trisomic patterns observed for all chromosome-specific STRs are indicative of triploidy. Therefore accurate X chromosome dosage, to perform diagnosis of X monosomy, can be assessed by TAF9L marker allowing This gene has a high degree of sequence identity between chromosome 3 and chromosome X; primers on this gene amplify a 3 b.p. deletion generating a chromosome X specific product of 141 b.p. and a chromosome 3 specific product of 144 b.p.