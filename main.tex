\documentclass[fleqn,12pt]{wlscirep}
\usepackage[utf8]{inputenc}
\usepackage[T1]{fontenc}
\usepackage{placeins}
\usepackage{graphicx}
\usepackage{lscape}
\usepackage{todonotes}
%\usepackage{xcolor}
\usepackage{multirow}
\usepackage[table,xcdraw]{xcolor}
\usepackage{lscape}

\newcommand{\beginsupplement}{%
        \setcounter{table}{0}
        \renewcommand{\thetable}{S\arabic{table}}%
        \setcounter{figure}{0}
        \renewcommand{\thefigure}{S\arabic{figure}}%
     }
\newcommand{\gp}[]{\textsc{gp }}

%\title{Prioritization for diagnosis through whole-genome sequencing of product of conception from idiopathic pregnancy losses }
\title{Prioritization of causative genomic variants in miscarried embryos from idiopathic pregnancy losses }

\author[1,2+]{Silvia Buonaiuto}
\author[3+]{Imma Di Biase}
\author[4]{Valentina Aleotti}
\author[5]{Adriano De Marino}
\author[1,6]{Gianluca Damaggio}
\author[7]{Marco Chierici}
\author[8]{Madhuri Pulijala}
\author[3]{Palmira D’Ambrosio} 
\author[3]{Gabriella Esposito}
\author[8]{Qasim Ayub}
\author[7]{Cesare Furlanello}
\author[9]{Nicole Soranzo}
\author[4]{Michele Rubini}
\author[5]{Antonio Capalbo}
\author[3]{Sebastiano Di Biase}
\author[1*]{Vincenza Colonna}
%\author[]{{\color{blue} nomi memorial exomes} }
\affil[1]{National Research Council, Institute of Genetics and Biophysics Adriano Buzzati-Traverso, Napoli, Italy}
\affil[2]{University of Campania Luigi Vanvitelli,Department of Environmental, Biological and Pharmaceutical Sciences and Technologies,Caserta, Italy}
\affil[3]{MeriGen Research s.r.l., Naples, Italy}
\affil[4]{University of Ferrara, Ferrara, Italy}
\affil[5]{Igenomix Italy, Marostica, Italy}
\affil[6]{University of Naples Federico II,Department of Biology, Naples, Italy}
\affil[7]{FBK-Fondazione Bruno Kessler, Povo (Trento), Italy}
\affil[8]{Monash University Malaysia Genomics Facility, School of Science, Bandar Sunway}
\affil[9]{Wellcome Sanger Institute, Hinxton, UK}
\affil[*]{Correspondence: vincenza.colonna@igb.cnr.it (V.C.)}
\affil[+]{these authors contributed equally to this work}


\begin{document}

\flushbottom
\maketitle
% * <john.hammersley@gmail.com> 2015-02-09T12:07:31.197Z:
%
%  Click the title above to edit the author information and abstract
%

%%%%%%%%%%%%%%%%%• Present the problem and the proposed solution %%%%%%%%%%%%%%%%%• Presents nature and scope of the problem investigated
\textbf{Background.} Miscarriage, i.e. the spontaneous termination of a pregnancy before 24 weeks of gestation, occurs in  10-15\% of all pregnancies and has both environmental and genetic causes.\cite{larsen2013new,ammon2012systematic, andersen2000maternal} Miscarriages are often caused by chromosomal aneuploidies of the gametes but they can also have non random genetic causes like small-size mutations both \textit{de-novo} or inherited. Miscarriages are mostly studied using parental genetic information\cite{pereza2017systematic, laisk2019genetic},  %and at a resolution that leaves unexplored the vast majority of the genome. %Comparative genomic hybridization detects variants of several thousand base pairs \cite{robberecht2009diagnosis, kudesia2014rescue,mathur2014miscarriage} while targeted resequencing resolves point mutations. Both are currently the most accurate methods for the genetic analysis of parental DNA of miscarriages but are not sensitive to small variants, or target only a few coding regions. 
%On a different approach, and the only genome-wide genetic association study is also based on maternal information.\cite{} 
%Depending on the mode of inheritance the study of parental genome 
which might be ineffective as it will reveal only half of the inheritance in the embryo and it would miss \textit{de novo} mutations. Extending therefore the analysis to fetal genomes is the necessary next step 
%to fully understand the genetics of miscarriages 
together with an approach that systematically targets also small-size genetic variants. 
%The incidence of genomic abnormalities in RPL is estimated to be around 50\% \cite{van2012genetics}. 

%%%%%%%%%%%%%%%%% • Reviews the pertinent literature to orient the reader
%%literature: miscarriages sequences 
DNA sequence information of miscarried fetuses has already been used to study genetic components of miscarriages.\cite{rajcan2020next, filges2015exome} Most studies on candidate genes\cite{bondeson2017nonsense, dohrn2015ecel1, wilbe2015musk, cristofoli2017novel, rae2015novel, thomas2012tctn3} or exomes\cite{shamseldin2015identification, qiao2016whole,fu2018whole, meier2019exome, yates2017whole} consider number of cases in the order of the tens and adopt a family-based approach often with focus on a reduced range of fetal phenotypes deriving from ultrasound scans. 
%, integrating pedigree and parental genomic data, and considering either candidate genes\cite{bondeson2017nonsense, dohrn2015ecel1, wilbe2015musk, cristofoli2017novel, rae2015novel, thomas2012tctn3} or whole-exome %Some among those focusing only on embryos target candidates genes. Examples are the identification of a mutation in the X-linked gene \textit{FOXP3} in siblings male miscarriages \cite{rae2015novel}, and the identification of a truncating \textit{TCTN3} mutations in unrelated embryos\cite{thomas2012tctn3}. A number of studies focuses instead on exome
%sequences.\cite{shamseldin2015identification, qiao2016whole,fu2018whole, meier2019exome, yates2017whole}% One study selects only variants transmitted to both sibling miscarriages \cite{qiao2016whole}, others limit to autozygous variants\cite{thomas2012tctn3, shamseldin2015identification}, some focus on delivering accurate diagnosis \cite{meier2019exome}.
%All these studies  and in most cases are motivated by phenotypic information deriving from ultrasound scans. 
%%%%%%%%%%%%%focus in idiopathic 
Two studies adopt a cohort-based approach analyzing up to thousands of embryonic genomes in a wide range of phenotypes with focus on evaluating the diagnostic potential of exome sequencing\cite{zhao2020exome} and identification of genes essential for the embryonic development.\cite{chen2017characterization}
%%literature: prioritization pipelines 
Because the number of embryos considered is too small for genetic association analysis to be effective, all studies perform sequencing followed by variant annotation and prioritization, investigating rare variation in supposedly euploid embryos. Nevertheless always different criteria are used to select the variants, and in any case the code to fully reproduce the variant prioritization is not available, making it difficult to replicate results and perform comparisons.


\textbf{Results.} To understand genetic susceptibility to miscarriage we studied the genome of 46 miscarried embryos %and nine women (not related to the forty-six embryos) experiencing spontaneous pregnancy termination. 
%The embryos gestational age at pregnancy termination, calculated as the interval between the pregnancy termination date and the last menstruation date, ranges from 7.14 to 19.43 weeks (median is 10.3 weeks). Twenty-one embryos classifies as the product of recurrent miscarriages \cite{eshre2018eshre}. The mothers of the embryos are mostly of European origin (87\%) and their median age at the date of collection was 36.7$\pm$5.9 years, with no significant difference between first and recurrent cases (Figure \ref{fig:embryostats}, Mann–Whitney p-value=0.02). For the mothers of the embryos medical records report no major comorbidities. Folic acid was taken by 71\% of the mothers with no difference between first and recurrent cases (Figure \ref{fig:embryostats}, Chi-square p-value=0.96). Median body mass index and menarche age are comparable between first and recurrent cases, as well as comparable to a group of control women (Figure \ref{fig:embryostats}). Altogether, from the available medical records, we suppose that the recruited 
from mothers in the range of healthy adult individuals. %The nine women with recurrent clinical miscarriage had an average of 4 previous clinical miscarriages (SD=2.7, range 3-11), average age of 33 years (SD=6; range 23-40), average BMI within the range of normality (24.6; SD=3.9), and ovarian reserve in the XX percentile by the mean age (Anti-Mullerian hormone mean 2.1 pmol/l; SD=2.1).
%It is known from literature that roughly xx\% of the miscarriages in the first trimester are due to large chromosomal aneuploidies, such as trisomies or deletions of large chromosomal chunks \cite{goddijn2000genetic,zhang2009genetic}. In this study we want to focus on cases in which the genome is euploid, therefore 
After screening for chromosomal aneuploidies, 32.6\% of samples were euploid and could be sequenced while 56.6\% of the embryos presented aneuploidies. %The most common aneuploidy in our data set is the trisomy of chromosome 22 (26.9\%), followed by trisomy of chromosome 16 (15.4\%). In particular, a first round of detection of aneuploidies on chromosomes 13, 15, 16, 18, 21, 22, X, and Y through Short Tandem Repeats analysis discarded 45.7\% of samples, %These types of repeats (tetra- or penta-nucleotide) are often expected to be found in heterozygosis, therefore triploidy is assumed when three alleles are found at several markers along a chromosome (complete) or part of it (partial). Similarly, uniparental disomy for a targeted region or chromosome is assumed when only one parental allele is amplified. %and a subsequent analysis through comparative genomic hybridization and copy number %variation detection form low-coverage sequencing discarded another 10.9\% of the samples. Finally, a number of embryos (10.9\%) dropped off the analysis due to low-quality DNA or maternal contamination.
%After ascertaining euploidy, the exome of the nine women was sequenced using Agilent SureSelect whole-exome capture and Illumina sequencing technology on the NovaSeq 6000 Series Sequencer(Next Generation Solutions, Hong Kong, China), while 
The whole genome of ten embryos was sequenced using Illumina short-reads at 30X coverage and 11M single-nucleotide polymorphisms (SNPs) and 2M small insertions or deletions were identified.% , while from the exome data of the women we identified 1.7M SNPs and 276k indels.
%In the set of embryos genomes, we identified 11041,557 single-nucleotide polymorphisms (SNPs), 1980256 small insertions or deletions (indels), and XX copy number variants (CNVs). In the set of women we identified 1738895 SNPs and 275652 indels.


%5493 genes in hgdp for exomes
%1323 discarded and 76% retained
%\subsection*{Prioritization of genetic variants in coding genomic regions} 
We developed \gp,\cite{gp2020} a pipeline to prioritize putatively causative variants in coding regions. \gp performs filtering of high-quality genomic variants based on functional predictions and using a set of parameters that can be specified by the user. After that, \gp filters for technical artifacts and for random selection of genes in a control cohort through resampling. 
\gp uses functional annotations of genomic variants, information from publicly available sequence data of presumably healthy individuals, and, if available, knowledge of genes involved in the trait under study. % \gp currently analyze coding regions and performs four filtering steps (Fig.\ref{fig:pipeline}). The first filter (Filter I) retains variants based on: (i) a user-defined threshold of allele frequency in control populations; (ii) an overall impact on the gene product classified as moderate or high [REF]; (iii) the combined property of being putatively damaging (quantified by the CADD score [REF]) and located in genes intolerant to loss of function (determined by the pLI score [REF]). In addition it is possible to incorporate one or more user-defined lists of genes relevant to the trait under study, and take this into account when applying Filter I. Variants retained by Filter I (hits) are further filtered to control for false positives with Filters II and III. Finally, Filter IV excludes private variants with read depth outside the range found in non-private ones.   
%Filter II removes variants in genes with too many hits. Filter III controls for the chance of random occurrence of genes based on replicates of Filter I analysis in a large control population. In practice it removes all the genes that pass Filter I a user-defined-fraction of times across a user-defined-number of replicates. 
We applied the \gp pipeline to data from the high-coverage whole-genome sequences of the embryos % the women (unrelated to embryos). For Filter I we set allele frequency <0.05\% in the 1000 Genomes\cite{1000genome2015global} and gnomAD\cite{lek2016analysis} reference populations, while the functional effect of the variant within the gene context was taken into account in two ways: either selecting for putatively deleterious variants (CADD score >90th percentile) in genes highly intolerant to loss of function (pLI score >0.9), or selecting for variants in genes known to be involved in early embryonic development. In particular for this last option we included five lists of genes, namely genes involved in embryo development (Gene Ontology GO:0009790), genes lethal during embryonic stages \cite{dawes2019gene}, essential for embryo development \cite{dawes2019gene}, genes discovered through the Deciphering Developmental Disorders project [REF DD], and a manually curated list of candidate genes known to be involved in miscarriages. We requested the variant to satisfy one or both these criteria: (i) be in a gene present in at least two of the five lists or (ii) have CADD score above the 90th percentile and be in a gene with pLI>0.9. Overall, filter I retained 1,038 and 728 variants (hits) in embryo and mothers, respectively.   
%Filter II removed variants in genes with >5 hits. With few exceptions, we observed that the majority of genes (99th percentile) has less than five hits, even if there is no significant correlation between number of hits and gene length (Spearman r=XX p-value=xx, Figure \ref{fig:filters}A). Genes with >5 hits have more paralogues than genes with <5\% (???) and hits in gene with >5 hits are enriched for private variants  We concluded that variants found in gene with more than five hits are likely to be sequencing and alignment artifacts and therefore are discarded. 
%For Filter III we 
using as control population 929 individuals from the Human Genome Diversity Project\cite{bergstrom2019insights} from which we resampled 100 times ten individuals.
%On each resampled set we performed Filter I analysis and recorded the genes that were retained. Overall 5,488 unique genes were retained in controls with different frequencies in samples across replicates (Figure \ref{fig:filters}B ). When considering the 95th percentile, 1,531 genes are found >5\% of times across replicates, therefore hits within these genes were removed by filter III. Because the criteria used for Filter I were identical between embryos and mothers, and because the numerosity was very similar, we performed resampling only once and applied the outcome to both analyses.
%Filter II and III, retained 447 and 359 hits in embryo and mothers, respectively of which 21\% in the embryos and 42\% in the women are private with respect to 1000 Genomes and gnomAD data sets. While in the embryos the read depth is comparable between private and non-privates variants, in women private variants have significantly lower coverage compared to non-private (Figure \ref{fig:filters}C-D, KS test p-value=XX, F test p-value=XXX). Therefore, to further control for possible artifacts due to scanty cverage, filter IV further removes hits which are private variants with read depth outside the range found in non-private ones.   

\gp prioritized 439 unique variants in 399 genes in the ten embryonic sequences. %(Fig.\ref{fig:resembryo}). %The genes prioritized in women are slightly enriched for genes belonging to mitochondrial translation initiation (R-HSA-5368286), elongation (R-HSA-5389840), and termination (R-HSA-5419276) pathways (p-value=0.0002, Benjamini-Hochberg adjusted p-value=0.0574, FDR=0.056). This enrichment is due to eight genes selected by \gp in six women due to missense mutations (Table \ref{tab:mito}).  --More about the eight genes %(\textit{MRPL15}, \textit{MRPL34}, \textit{MRPL37}, \textit{MRPS15}, \textit{MRPS21}, \textit{MRPS23}, \textit{MRPS28}, and \textit{OXA1L}) 
%%%%%%%%%%%%%%%%%%%%%%%%%%%%%%%%%%%%%%%%%%%%%%%%%%%%%
%\textit{Properties and biological significance of the prioritized variants and genes}. 
%\subsubsection*{Mutations in \textit{STAG2},\textit{TLE4}, \textit{FMNL2}, and \textit{FRMPD3} in the embryos} 
4.1\% of the prioritized variants % stop gains/loss, frameshift indels, and variants that disrupt splicing sites, all 
classify as having high impact on the gene products, while missense mutations prevail among the variant with moderate effect. 18.8\% of the prioritized genes are not in the lists of candidate genes used by \gp as input during the prioritization, demonstrating that \gp is robust to detection of genes never investigated before.% in relationship to the phenotype under study. 
%Averages per embryos are 48.9$\pm$8.0 genomic variants in 47.8$\pm$7.7 genes coding for 113.5$\pm$24.6 transcripts (Figure \ref{fig:resembryo}B). %For almost all genes \gp retains one variant per embryo, with few exception (five cases with two e and one with three variants per gene), as shown in Figure \ref{fig:resembryo}B, where the allele dosage and impact are also shown. %--Shared genes/variants 

We identified genetic variants in genes involved in embryonic development. Among them   
%validation merigen? 
%The male embryo AS030 carries 
one extremely rare high-impact alteration of a splice site in \textit{STAG2}, coding %(rs913664484, G frequency is 4.7e-05 in 42.7k individuals from gnomeAD\cite{lek2016analysis})that changes one of the two basis at the 5' end of the first intron of the 
for the cohesin complex subunit\cite{cuadrado2020specialized,mcnicoll2013cohesin}  for which  % (high impact). 
%\textit{STAG2} %is on the X chromosome and 
inactivation in mouse is lethal\cite{de2020essential} and only mildly-deleterious mutations are seen in living human males.\cite{mullegama2019mutations}  
%cohesin subunit SA-2 \cite{cuadrado2020specialized}, a ring-shaped protein complex that brings into close proximity two different DNA molecules or two distant parts of the same DNA molecule \cite{mcnicoll2013cohesin}. In mouse, inactivation \textit{Stag2} causes early embryo lethality \cite{de2020essential}, and % mutations in \textit{STAG2} have been discovered in children with developmental disorders (REF DDD). Interestingly, 
%only mildly-deleterious mutations have been found in living males \cite{mullegama2019mutations}. 
%The embryo AS036 carries a
We also found an heterozygous missense mutation in the \textit{TLE4} % (ENSG00000106829, synonym \textit{GRG-4})
gene, % that %causes a substitution of a polar amino acid with another polar amino acid (Ser>Tre) in the 7 exon of the gene, corresponding to a low complexity domain of the protein. The rs41307447 polymorphism is tolerated (SIFT score 0.18) and supposed to be benign (PolyPhen score 0.003), nevertheless the \textit{TLE4} gene 
%is classified as highly intolerant to loss of function (pLI score 0.999).% and the CADD score associated to rs41307447 is in the 99.8th percentile. 
%\textit{TLE4} is 
a trascriptional repressor of
%the Groucho-family expressed in the 
embryonic stem cells % where it repress 
naive pluripotency as target of Notch,\cite{menchero2019transitions, laing2015gro} also expressed in the extravillous trophoblasts as part of the Wnt pathway,% that promotes implantation, trophoblast invasion, and endo-metrialfunction
\cite{sonderegger2010wnt, meinhardt2014wnt} physically interacting with a region on chromosome 9 associated to miscarriages.\cite{laisk2019genetic}  
%recent association study in a cohort of 750 cases of
%women experiencing recurrent miscarriages found a signal on chromosome 9 %(rs7859844) 
%in a genomic region physically interacting with \textit{TLE4} , corroborating the hypothesis of association between variation in \textit{TLE4} and miscarriages. 

% AS093, AS090, AS087, AS065, AS036) 
%Five embryos share one copy of an haplotype composed of two T alleles 4bp apart (rs866373641, rs750755379) in the \textit{FMNL2} gene% (Table \ref{tab:fmnl2}). 
%The two alleles exists at moderate-to high frequency in human populations% (Table \ref{tab:fmnl2})
%and are in perfect linkage disequilibrium (r\textsuperscript{2}=1) in the embryos, as well as in other cohorts (r\textsuperscript{2}= 1 in HGDP).% hgdp etc..). 


\textbf{Future perspective.} We are interested in understanding genetic causes of miscarriages in a larger cohort of embryo samples under collection. In this pilot we developed a robust approach to investigate coding regions. The next logical extension will be to non-coding genomic regions, as well as to a more comprehensive set of variation, including copy number and structural variants. 



%\section*{Discussion}

1. Begin with a restatement of your research question, followed by a statement about whether or not, and how much, your findings "answer" the question.  These should be the first two pieces of information the reader encounters.


Here we want to understand the requirements for large scale population-based study of genomic sequences of unrelated miscarriages focused at dechipering the contribution of small-size mutation 

provided parents with an explanation of the developmental abnormality, delineated the recurrence risks, and assisted the management of subsequent pregnancies.



modelli dominante/recessivo /de novo 


 However, sequencing is usually done at very low resolution and limited to exomes, i.e. 2\% of the genome, leaving unexplored the vast majority of the genome. 
 
 
 %%%literature: other sequences 
Analysis of genetic variants from exome data improves the genetic diagnosis of fetal structural anomalies when standard investigations (karyotype testing and chromosomal array) are uninformative, as shown by studies on hundreds of trios in wide ranges of gestational ages, phenotypes detected by ultrasound, and pregnancy outcomes, including livebirths \cite{petrovski2019whole, quinlan2019molecular, lord2019prenatal}. 



%Future prspectives: Calibration? integration of gene expression? non-coding? positive control? Copy number variants ? 

%Despite the its decreasing costs, whole-genome sequencing is not yet applied to the diagnosis of aneuploidies  ... \\
%Rare variants have large effects, natural selection prevent them to become common 
%We developed a pipeline to select cases of PLs in which the genome of the PoC is euploid and the mother does not present obvious comorbidities. These cases are similar to cases of idiopathic miscarriages that can be used to target the identification of small-size lethal genomic variants through whole genome sequencing.\\

%The identification of small variants requires large sample size. We observe the fraction of samples which... therefore we estimate that the number of samples to collect shuold be  X times the number of samples to be sequenced ...  a sample size of ...  is required to .... Figure \ref{fig:fractions}\\

%We also learned something about miscarriages: report aggregate statistics of qfPCR and arrayCGH when will be available.\\ 

%samples not used for sequencing can be used to study chromosomal rearrangements 

%Limitations: \\
%- array CGH: oinversion not visible.  only deletion and duplication but when complex it is impossible to determine the  order of the fragments. Complex chromosomal rearrangements  and Chromoanagenesis that do not involve copy nuber variants can not be identified.\\ 
%- Is it valid price-wise or better do low-coverage sequencing? 

%\section*{Methods}

\subsection*{Women sample and data collection}
Text here 

\subsection*{Embryo data and samples collection}
Samples collection was done by the Unit of Obstetrics and Gynecology of the Sant’Anna University Hospital in Ferrara, Italy, from 2016 to 2020. It was approved by the local Ethical Committee (approval number CE/FE 170475) and carried out in compliance with the Helsinki Declaration. All participants provided written informed consents before recruitment. The inclusion criteria were: age in the range 18–42 years; gestational age within the first 12 weeks. Maximum gestational age for cases of voluntary termination of pregnancy was ninety days, according to the Italian law, namely Bill 194, Article 4. 


Anonymous data about age, body mass index, menarche age, previous pregnancies, and geographical origin were considered for this study. Data cleaning, refining, and analysis (summary statistics, hypothesis testing) were performed using R \cite{R}.


Fresh embryonic tissues were analyzed the same day of collection. Chorionic villi (CV) were separated from the maternal decidua in sterility under a hood using a stereomicroscope (Leica Microsystems Srl, All Microscopy and Histology, Milan, I-20142 Italy). The villi were stored at -20\textdegree  C for a few months or at 4\textdegree  C in RPMI media for not more than a week before proceeding to homogenization and DNA extraction. We explored a range of possibilities for DNA preparation from CV that includes two methods of tissue homogenization and three methods of DNA isolation. We do not observe significant difference between homogenization techniques, therefore we proceeded with dry homogenization that is technically less challenging (Figure \ref{fig:dna}A). Similarly, in the case of DNA isolation we considered two types of resin (Valentina) and one membrane (Valentina) and we do not observe significant differences in yield neither among the techniques or among samples from maternal blood, voluntary termination of pregnancy and miscarriages but slightly higher range of yield for one type of resin (Figure \ref{fig:dna}B and Figure \ref{fig:dnayeld}). Quantification of genomic DNA was done with Qubit® 2.0 Fluorometer (Invitrogen) according to manufacture instructions.

%Genomic DNA (gDNA) was extracted from chorionic villi dissected from abortion tissue specimens using QIAamp DNA Blood Mini Kit (Qiagen), and with Instagene TM Matrix (Bio-Rad) according to manufacturer protocols \textit{(QIAamp DNA Mini and Blood Mini Handbook 05/2016. Instruction Manual, InstaGene™ Matrix, LIT544 Rev G)}

--Maternal contamination XXXX 
%TBA  \todo{MERIGEN}

\subsection*{Detection of chromosomal aneuploidies in embryos} 
\subsubsection*{Detection of sex and numerical anomalies through quantitative PCR}
A rapid screening of sex and numerical anomalies for chromosomes 13, 15, 16, 18, 21, 22 and X was carried out with the miscarriage DNA samples performing five multiplex PCR assays. PCR assays A, B 1,2 and 3 were performed in a total volume of 25μl containing 40–100ng of genomic DNA, 10mM dNTP (Roche), 6-30 pmol (final concentration) of each primer, 1×Fast taq polymerase buffer (15mmol/l MgCl 2 ) (Roche), and 2.5 U of Fasta taq polymerase (Roche). PCR conditions were as follows: denaturation at 95\textdegree C for 10 min followed by 10 cycles consisting of melting at 95\textdegree C for 1 min, annealing at 65\textdegree C (-1\textdegree C / cycle) for 1 minutes, and then extension at 72\textdegree C for 40 seconds, then for 23 cycles at 95\textdegree C for 1 min, 55\textdegree C for 1 min, and 72\textdegree C for 1 min. Final extension was for 10min at 72\textdegree C and at 60\textdegree C for 60 min. Fluorescence-labelled PCR products were electrophoresed in an CEQ 8000 Backman by combining 40 μl of Hi-Di Formamide and 0.5 μl of DNA size standard 400 (Backman); The resulting PCR products can be visualized and quantified as peak areas of the respective repeat lengths. In normal heterozygous subjects, the QF- PCR product of each STR should show two peaks with similar fluorescent activities and thus a ratio of peak areas close to 1:1 (ranging from 0.8 to 1.4:1). A trisomy is suspected when the ratio is  above or below this range (peak area ratios ≤ 0.6 and ≥ 1.8) (trisomic diallelic);otherwise there are three alleles of equal peak area with a ratio of 1:1:1 (trisomictriallelic). The presence of trisomic triallelic or diallelic patterns for at least two different STRs on the same chromosome is considered as evidence of trisomy. Trisomic patterns observed for all chromosome-specific STRs are indicative of triploidy. Therefore accurate X chromosome dosage, to perform diagnosis of X monosomy, can be assessed by TAF9L marker allowing This gene has a high degree of sequence identity between chromosome 3 and chromosome X; primers on this gene amplify a 3 b.p. deletion generating a chromosome X specific product of 141 b.p. and a chromosome 3 specific product of 144 b.p.

\subsubsection*{Comparative Genomic Hybridization} 
--Imma 
Agilent SurePrint G3 CGH-only. Log-ratio produced by the Agilent CytoGenomics v3.0.4 software 

Beside the Agilent software XXX , data were analyzed using custom R scripting (Figure \ref{fig:cnvmethods}). The LogRatio from the arrays were segmented into regions of estimated equal copy number using both the method implemented in the Agilent, and the Penalized least square implemented in the copynumber R package (PLS, \cite{nilsen2012copynumber}) and classified as copy number of gains or losses (copy number variants, CNV) using as criteria at least five probes and Zscore <0.0016 (SD*4)\cite{vermeesch2005molecular}. 
 %Circular Binary Segmentation (CBS)\cite{Venkatraman2018}, 



\subsection*{DNA sequencing and sequence analysis}
Whole-genome sequencing of the embryonic DNA was done through a service provider (Macrogen s.r.l). In particular, libraries for sequencing were prepared using the Illumina TruSeq DNA PCR-free Library (insert size 350bp) and samples were sequenced at 30X mapped (~110Gb) 150bp PE on HiSeqX. Data were released in as fastq files. -- Exome sequencing of the women XXXXXX
%SILVIA We obtained XX GB of data per individual, with.95.8 of the genome covered more than XX times.

Reads in the fastq file of both embryo and women sequence data were aligned against the reference genome GRChg38.p12 \cite{rosenbloom2015ucsc} using \textsc{bwa}\cite(Li 2009){} and \textsc{samtools}\cite{}. Variant calling was done using \textsc{freebayes} \cite{garrison2012haplotype}. The resulting vcf files were refined in three steps: (i) \textsc{vcffilter} \cite{vcflib} was used to filter variants for quality score>20, leaving only variants with 99\% accuracy of genotype call; (ii) \textsc{vt} \cite{tan2015unified} was used to normalize variants; (iii) \textsc{vt}  \cite{tan2015unified} was used to deconstruct multiallelic variants. Refined vcf files were compressed and indexed using samtools \cite{}. Variants were annotated for functional effects and allele frequency in other populations using Variant Effect Predictor (VEP, McLaren2016).  

The \gp pipeline for variant prioritization is written in Python and the code is publicly avaiable (\url{https://github.com/ezcn/grep}). 
 
The computational work has been executed on the IT resources of the ReCaS-Bari data center, which have been made available by two projects financed by the MIUR (Italian Ministry for Education, University and Re-search) in the "PON Ricerca e Competitività 2007-2013" Program: ReCaS (Azione I - Interventi di rafforzamento strutturale, PONa3\_00052, Avviso 254/Ric) and PRISMA (Asse II - Sostegno all'innovazione, PON04a2\_A)"


%Variant calling, phasing and annotation 
%Raw paired end sequencing data was mapped on the reference genome  GRCh38 using BWA (Li 2009) and variant calling was done using Freebayes (Garrison 2012), discovering on average 436k variants per sample with Phred-scaled quality scores >20. Information on functional annotations and Combined Annotation Dependent Depletion (CADD) for variable sites was obtained using Variant Effect Predictor (McLaren 2016). Phasing was done using Beagle 5.1 (Browning 2018) under standard parameters.  


\subsection*{PCA Analysis, Reactome} 
%Approximately 15,000 autosomal exonic SNPs from HapMapPhase 325were used to conduct PCA analysis. SNPs were prunedusing PLINK26based on LD (variance inflation factor thresholdof 2), and only common variants (MAF>5\%) were used in thePCA analysis. PCA outlier analyses were performed by projectingthe case samples onto HapMap3 samples using EIGENSOFT




\bibliography{sample}
\section*{Keywords}
miscarriages, variant prioritization, whole-genome DNA sequence
\section*{Contacts}
vincenza.colonna@igb.cnr.it\\
silvia.buonaiuto@igb.cnr.it
\section*{Websites}
\url{https://colonnalab.github.io/laboratory_WebPage/}
\url{https://github.com/ezcn/grep}

%\FloatBarrier
%
\begin{figure}[h]
\centering
\includegraphics[width=0.7\textwidth]{fig/ibelieve.png}
\caption{\textbf{Pre-sequencing screening outcome.} }
\label{fig:presequencing}
\end{figure}

\begin{figure}[ht]
\centering
\includegraphics[width=\linewidth]{fig/pipeonly.png}
\caption{\textbf{Overview of the pipeline for prioritization for sequencing (A) Main steps.} Samples are first screened for the quality of DNA and maternal contamination and then analyzed for aneuploidies. \textbf{(B) Outcomes of the pipeline.} We estimate that 18\% of samples goes to sequencing....} 
\label{fig:pipeline}
\end{figure}

\begin{figure}[ht]
\centering
\includegraphics[width=\linewidth]{fig/filters_EmbryoWomen.png}
\caption{\textbf{}}
\label{fig:filters}
\end{figure}

\begin{figure}[ht]
\centering
\includegraphics[width=\linewidth]{fig/panel_EmbryoResults.png}
%\caption{\textbf{} he occurrence of variants, their impact, and the count of the consequence allele per gene per embryo  }
\caption{\textbf{}}
\label{fig:resembryo}
\end{figure}

\begin{figure}[ht]
\centering
\includegraphics[width=\linewidth]{fig/panel_WomenResults.png}
\caption{\textbf{}  }
\label{fig:reswomen}
\end{figure}



%\FloatBarrier
%\begin{landscape}
\begin{table}[]
\resizebox{\textwidth}{!}{%
\begin{tabular}{|l|l|l|l|l|l|l|l|}
\hline
{\color[HTML]{000000} \textbf{Individual}}      & {\color[HTML]{000000} \textbf{Gene}}            & {\color[HTML]{000000} \textbf{Location}}        & {\color[HTML]{000000} \textbf{Consequence}}       & {\color[HTML]{000000} \textbf{Transcript}} & {\color[HTML]{000000} \textbf{Substitution description}}   & {\color[HTML]{000000} \textbf{SIFT (score)}}             & \textbf{PolyPhen (score)}         \\ \hline
{\color[HTML]{000000} }                         & {\color[HTML]{000000} }                         & {\color[HTML]{000000} }                         & {\color[HTML]{000000} }                           & {\color[HTML]{000000} ENST00000260102}     & {\color[HTML]{000000} c.752G\textgreater{}A, p.Ala237Thr}  & {\color[HTML]{000000} deleterious (0)}                   & deleterious (0)                   \\ \cline{5-8} 
{\color[HTML]{000000} }                         & \multirow{-2}{*}{{\color[HTML]{000000} MRPL15}} & \multirow{-2}{*}{{\color[HTML]{000000} chr 8}}  & \multirow{-2}{*}{{\color[HTML]{000000} missense}} & {\color[HTML]{000000} ENST00000522521}     & {\color[HTML]{000000} c.532G\textgreater{}A, p.Ala178Thr}  & {\color[HTML]{000000} deleterious (0)}                   & deleterious (0)                   \\ \cline{2-8} 
\multirow{-3}{*}{{\color[HTML]{000000} 191544}} & {\color[HTML]{000000} MRPS15}                   & {\color[HTML]{000000} chr 1}                    & {\color[HTML]{000000} missense}                   & {\color[HTML]{000000} ENST00000373116}     & {\color[HTML]{000000} c.864G\textgreater{}A, p.Thr252Ile}  & {\color[HTML]{000000} tolerated (0.1)}                   & tolerated (0.1)                   \\ \hline
{\color[HTML]{000000} }                         & {\color[HTML]{000000} }                         & {\color[HTML]{000000} }                         & {\color[HTML]{000000} }                           & {\color[HTML]{000000} ENST00000276585}     & {\color[HTML]{000000} c.154C\textgreater{}G, p.Arg48Pro}   & {\color[HTML]{000000} tolerated (0.06)}                  & tolerated (0.06)                  \\ \cline{5-8} 
{\color[HTML]{000000} }                         & {\color[HTML]{000000} }                         & {\color[HTML]{000000} }                         & {\color[HTML]{000000} }                           & {\color[HTML]{000000} ENST00000518271}     & {\color[HTML]{000000} c.127C\textgreater{}G, p.Arg43Pro}   & {\color[HTML]{000000} tolerated (0.06)}                  & tolerated (0.06)                  \\ \cline{5-8} 
{\color[HTML]{000000} }                         & \multirow{-3}{*}{{\color[HTML]{000000} MRPS28}} & \multirow{-3}{*}{{\color[HTML]{000000} chr 8}}  & \multirow{-3}{*}{{\color[HTML]{000000} missense}} & {\color[HTML]{000000} ENST00000521605}     & {\color[HTML]{000000} c.184C\textgreater{}G, p.Arg48Pro}   & {\color[HTML]{000000} deleterious (0)}                   & deleterious (0)                   \\ \cline{2-8} 
{\color[HTML]{000000} }                         & {\color[HTML]{000000} }                         & {\color[HTML]{000000} }                         & {\color[HTML]{000000} }                           & {\color[HTML]{000000} ENST00000594999}     & {\color[HTML]{000000} c.351C\textgreater{}T, p.Ala74Val}   & {\color[HTML]{000000} deleterious (0.02)}                & deleterious (0.02)                \\ \cline{5-8} 
{\color[HTML]{000000} }                         & {\color[HTML]{000000} }                         & {\color[HTML]{000000} }                         & {\color[HTML]{000000} }                           & {\color[HTML]{000000} ENST00000595444}     & {\color[HTML]{000000} c.529C\textgreater{}T, p.Ala166Val}  & {\color[HTML]{000000} deleterious low confidence (0.01)} & deleterious low confidence (0.01) \\ \cline{5-8} 
{\color[HTML]{000000} }                         & {\color[HTML]{000000} }                         & {\color[HTML]{000000} }                         & {\color[HTML]{000000} }                           & {\color[HTML]{000000} ENST00000600434}     & {\color[HTML]{000000} c.571C\textgreater{}T, p.Ala74Val}   & {\color[HTML]{000000} deleterious (0.02)}                & deleterious (0.02)                \\ \cline{5-8} 
\multirow{-7}{*}{{\color[HTML]{000000} 193012}} & \multirow{-4}{*}{{\color[HTML]{000000} MRPL34}} & \multirow{-4}{*}{{\color[HTML]{000000} chr 19}} & \multirow{-4}{*}{{\color[HTML]{000000} missense}} & {\color[HTML]{000000} ENST00000252602}     & {\color[HTML]{000000} c.446C\textgreater{}T, p.Ala74Val}   & {\color[HTML]{000000} deleterious (0.02)}                & deleterious (0.02)                \\ \hline
{\color[HTML]{000000} }                         & {\color[HTML]{000000} }                         & {\color[HTML]{000000} }                         & {\color[HTML]{000000} }                           & {\color[HTML]{000000} ENST00000605337}     & {\color[HTML]{000000} c.206C\textgreater{}T, p.Pro53Leu}   & {\color[HTML]{000000} deleterious low confidence (0)}    & deleterious low confidence (0)    \\ \cline{5-8} 
\multirow{-2}{*}{{\color[HTML]{000000} 181853}} & \multirow{-2}{*}{{\color[HTML]{000000} MRPL37}} & \multirow{-2}{*}{{\color[HTML]{000000} chr 1}}  & \multirow{-2}{*}{{\color[HTML]{000000} missense}} & {\color[HTML]{000000} ENST00000360840}     & {\color[HTML]{000000} c.235C\textgreater{}T, p.Pro53Leu}   & {\color[HTML]{000000} deleterious (0)}                   & deleterious (0)                   \\ \hline
{\color[HTML]{000000} }                         & {\color[HTML]{000000} }                         & {\color[HTML]{000000} }                         & {\color[HTML]{000000} }                           & {\color[HTML]{000000} ENST00000581066}     & {\color[HTML]{000000} c.670T\textgreater{}G, p.Leu75Val}   & {\color[HTML]{000000} tolerated (0.28)}                  & tolerated (0.28)                  \\ \cline{5-8} 
\multirow{-2}{*}{{\color[HTML]{000000} 192970}} & \multirow{-2}{*}{{\color[HTML]{000000} MRPS21}} & \multirow{-2}{*}{{\color[HTML]{000000} chr 1}}  & \multirow{-2}{*}{{\color[HTML]{000000} missense}} & {\color[HTML]{000000} ENST00000614145}     & {\color[HTML]{000000} c.434T\textgreater{}G, p.Leu75Val}   & {\color[HTML]{000000} tolerated (0.28)}                  & tolerated (0.28)                  \\ \hline
{\color[HTML]{000000} }                         & {\color[HTML]{000000} }                         & {\color[HTML]{000000} }                         & {\color[HTML]{000000} }                           & {\color[HTML]{000000} ENST00000313608}     & {\color[HTML]{000000} c.210G\textgreater{}T, p.Pro59His}   & {\color[HTML]{000000} tolerated (0.18)}                  & tolerated (0.18)                  \\ \cline{5-8} 
{\color[HTML]{000000} }                         & {\color[HTML]{000000} }                         & {\color[HTML]{000000} }                         & {\color[HTML]{000000} }                           & {\color[HTML]{000000} ENST00000578444}     & {\color[HTML]{000000} c.210G\textgreater{}T, p.Pro59His}   & {\color[HTML]{000000} tolerated (0.12)}                  & tolerated (0.12)                  \\ \cline{5-8} 
\multirow{-3}{*}{{\color[HTML]{000000} 190732}} & \multirow{-3}{*}{{\color[HTML]{000000} MRPS23}} & \multirow{-3}{*}{{\color[HTML]{000000} chr 17}} & \multirow{-3}{*}{{\color[HTML]{000000} missense}} & {\color[HTML]{000000} ENST00000579380}     & {\color[HTML]{000000} c.350G\textgreater{}T, p.Pro7Thr}    & {\color[HTML]{000000} -}                                 & -                                 \\ \hline
{\color[HTML]{000000} }                         & {\color[HTML]{000000} }                         & {\color[HTML]{000000} }                         & {\color[HTML]{000000} }                           & {\color[HTML]{000000} ENST00000285848}     & {\color[HTML]{000000} c.1131G\textgreater{}T, p.Met377Ile} & {\color[HTML]{000000} tolerated (0.51)}                  & tolerated (0.51)                  \\ \cline{5-8} 
{\color[HTML]{000000} }                         & {\color[HTML]{000000} }                         & {\color[HTML]{000000} }                         & {\color[HTML]{000000} }                           & {\color[HTML]{000000} ENST00000358043}     & {\color[HTML]{000000} c.1221G\textgreater{}T, p.Met301Ile} & {\color[HTML]{000000} tolerated (0.33)}                  & tolerated (0.33)                  \\ \cline{5-8} 
{\color[HTML]{000000} }                         & {\color[HTML]{000000} }                         & {\color[HTML]{000000} }                         & {\color[HTML]{000000} }                           & {\color[HTML]{000000} ENST00000412791}     & {\color[HTML]{000000} c.951G\textgreater{}T, p.Met317Ile}  & {\color[HTML]{000000} tolerated (0.37)}                  & tolerated (0.37)                  \\ \cline{5-8} 
\multirow{-4}{*}{{\color[HTML]{000000} 192152}} & \multirow{-4}{*}{{\color[HTML]{000000} OXA1L}}  & \multirow{-4}{*}{{\color[HTML]{000000} chr 14}} & \multirow{-4}{*}{{\color[HTML]{000000} missense}} & {\color[HTML]{000000} ENST00000612549}     & {\color[HTML]{000000} c.965G\textgreater{}T, p.Met317Ile}  & {\color[HTML]{000000} tolerated (0.28)}                  & tolerated (0.28)                  \\ \hline
\end{tabular}%
}
\caption{mito PL}
\label{tab:mito}
\end{table}
\end{landscape}


\begin{landscape}
\begin{table}[]
\resizebox{\textwidth}{!}{%
\begin{tabular}{|l|l|l|l|l|}
\hline
{\color[HTML]{000000} \textbf{rs identifier}} & \textbf{genomic location (GRCh38)} & \textbf{HGVS Names}  & \textbf{Functional annotations}                        & \textbf{Alternate allele frequency}\\ \hline
{\color[HTML]{000000} rs866373641}  & chr2 152560918  & c.479G\textgreater{}T, p.Ser160Ile  & missense, tolerated (SIFT=0.14), benign(PolyPhen=0.02) & 1000G=1\%, gnomADe=1\%, gnomADg=8\%, TOPMed=18\% \\ \hline
{\color[HTML]{000000} rs750755379} & chr2 152560914 & c.475G\textgreater{}T,  p.Glu159Ter & stop gain & 1000G=1\%, gnomADe=1.3\%, gnomADg=10.3\%, TOPMed=25.3\% \\ \hline
\end{tabular}%
}
\caption{\textit{FMNL2} TT haplotype - HGVS - Human Genome Variation Society}
\label{tab:fmnl2}
\end{table}
\end{landscape}

%\beginsupplement
%\section*{Supplementary Information}
%%%%%%%%%%%%%%%%%%%%%%%%%
%\subsection*{Supplementary Tables}
%\begin{landscape}
\begin{table}[]
\resizebox{\textwidth}{!}{%
\begin{tabular}{|l|l|l|l|l|l|l|}
\hline
{\color[HTML]{000000} \textbf{ID}} & \textbf{origin} & \textbf{Full term births} & \textbf{Previous miscarriages} & \textbf{Induced abortion} & \textbf{Preterm birth} & \textbf{type of miscarriage} \\ \hline
{\color[HTML]{000000} AS006}       & European        & 0                         & 0                              & 0                         & 0                      & miscarriage first            \\ \hline
{\color[HTML]{000000} AS030}       & European        & 1                         & 1                              & 0                         & 0                      & miscarriage first            \\ \hline
AS036                              & European        & 0                         & 0                              & 0                         & 0                      & miscarriage first            \\ \hline
AS054                              & African         & 1                         & 0                              & 2                         & 0                      & miscarriage first            \\ \hline
AS064                              & European        & 0                         & 0                              & 0                         & 0                      & miscarriage\_first           \\ \hline
AS065                              & European        & 0                         & 0                              & 0                         & 0                      & miscarriage first            \\ \hline
AS087                              & Asian           & 0                         & 3                              & 0                         & 1                      & miscarriage recurrent        \\ \hline
AS090                              & European        & 1                         & 0                              & 0                         & 0                      & miscarriage first            \\ \hline
AS093                              & Asian           & 0                         & 2                              & 0                         & 0                      & miscarriage recurrent        \\ \hline
AS094                              & European        & 3                         & 3                              & 0                         & 0                      & miscarriage recurrent        \\ \hline
\end{tabular}%
}
\caption{}
\label{tab:medicalfields}
\end{table}
\end{landscape}

\begin{landscape}
\begin{table}[]
\resizebox{\textwidth}{!}{%
\begin{tabular}{|l|l|l|l|l|}
\hline
\textbf{Gene symbol} & \textbf{Gene length (kb)} & \textbf{Nb. of paralogues (range identity)} & \textbf{hist in embryos} & \textbf{hits in women} \\ \hline
\textit{C2CD3}       & 7.96                      &                                             & 7                        & 0                      \\ \hline
\textit{FIGN}        & 9.5                       & 10 (24-41\%)                                & 16                       & 0                      \\ \hline
\textit{GXYLT1}      & 63                        &                                             & 20                       & 19                     \\ \hline
\textit{MTCH2}       & 29.8                      &                                             & 14                       & 12                     \\ \hline
\textit{MUC1}        & 7.09                      &                                             & 9                        & 9                      \\ \hline
\textit{MUC5B}       & 39.1                      &                                             & 0                        & 6                      \\ \hline
\textit{PCSK5}       & 472                       & 9 (11-52\%)                                 & 0                        & 6                      \\ \hline
\end{tabular}%
}
\caption{Caption}
\label{tab:paralogs}
\end{table}
\end{landscape}


%%%%%%%%%%%%%%%%%%%%%%%%%%
%\subsection*{Supplementary Figures}
%\begin{figure}[ht]
\centering
\includegraphics[width=\linewidth]{fig/panel_stats.png}
\caption{\textbf{Features of the embryo's mothers.} \textbf{(A)} Median age of the mother at the event is XX and XX for first and recurrent miscarriages, with no significant difference. \textbf{(B)} Gestational age at the time of the pregnancy termination range from X to X weeks with no significant difference between first and recurrent cases.  \textbf{(C)} Folic acid intake. Range of values of menarche age \textbf{(D)} and Body Mass Index \textbf{(E)} in embryo's mothers are not significantly different from a control set of mothers undergoing voluntary termination of pregnancy.}
\label{fig:embryostats}
\end{figure}

\begin{figure}[ht]
    \centering
    \includegraphics[width= 14 cm, high= 16cm]{fig/panelDNA.png}
    \caption{\textbf{Optimization of tissue homogenization and DNA extraction.} We do not observe significant difference between two methods of tissue homogenization (\textbf{A}), and three methods of DNA isolation (\textbf{b}) apart form a slightly higher range of yield for one type of resin. VTP: voluntary pregnancy termination;  FPL: first pregnancy loss; RPL:recurrent pregnancy loss;  MAT: maternal bllod; PoC: product of conception.}
    \label{fig:dna}
\end{figure}

\begin{figure}[ht]
    \centering
    \includegraphics[width= 14 cm, high= 16cm]{fig/totaldna_bykit.png}
    \caption{\textbf{Yeld of DNA extraction form chorionic villi by extraction kit.} Distributions of total DNA as quantified using the  Qubit 2.0 Fluorometer} 
    \label{fig:dnayeld}
\end{figure}

\begin{figure}[ht]
    \centering
    \includegraphics[width= 14 cm, high= 16cm]{fig/cnvCallComparison.png}
    \caption{\textbf{Copy number variant detection.} Comparison of calls made by the Agilent software and the penalized least square method implemented in the copynumber R package \cite{nilsen2012copynumber}} 
    \label{fig:cnvmethods}
\end{figure}


\end{document}