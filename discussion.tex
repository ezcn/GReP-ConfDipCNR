\section*{Discussion}

1. Begin with a restatement of your research question, followed by a statement about whether or not, and how much, your findings "answer" the question.  These should be the first two pieces of information the reader encounters.


Here we want to understand the requirements for large scale population-based study of genomic sequences of unrelated miscarriages focused at dechipering the contribution of small-size mutation 

provided parents with an explanation of the developmental abnormality, delineated the recurrence risks, and assisted the management of subsequent pregnancies.



modelli dominante/recessivo /de novo 


 However, sequencing is usually done at very low resolution and limited to exomes, i.e. 2\% of the genome, leaving unexplored the vast majority of the genome. 
 
 
 %%%literature: other sequences 
Analysis of genetic variants from exome data improves the genetic diagnosis of fetal structural anomalies when standard investigations (karyotype testing and chromosomal array) are uninformative, as shown by studies on hundreds of trios in wide ranges of gestational ages, phenotypes detected by ultrasound, and pregnancy outcomes, including livebirths \cite{petrovski2019whole, quinlan2019molecular, lord2019prenatal}. 



%Future prspectives: Calibration? integration of gene expression? non-coding? positive control? Copy number variants ? 

%Despite the its decreasing costs, whole-genome sequencing is not yet applied to the diagnosis of aneuploidies  ... \\
%Rare variants have large effects, natural selection prevent them to become common 
%We developed a pipeline to select cases of PLs in which the genome of the PoC is euploid and the mother does not present obvious comorbidities. These cases are similar to cases of idiopathic miscarriages that can be used to target the identification of small-size lethal genomic variants through whole genome sequencing.\\

%The identification of small variants requires large sample size. We observe the fraction of samples which... therefore we estimate that the number of samples to collect shuold be  X times the number of samples to be sequenced ...  a sample size of ...  is required to .... Figure \ref{fig:fractions}\\

%We also learned something about miscarriages: report aggregate statistics of qfPCR and arrayCGH when will be available.\\ 

%samples not used for sequencing can be used to study chromosomal rearrangements 

%Limitations: \\
%- array CGH: oinversion not visible.  only deletion and duplication but when complex it is impossible to determine the  order of the fragments. Complex chromosomal rearrangements  and Chromoanagenesis that do not involve copy nuber variants can not be identified.\\ 
%- Is it valid price-wise or better do low-coverage sequencing? 
